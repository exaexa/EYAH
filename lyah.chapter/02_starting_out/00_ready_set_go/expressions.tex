\exercise{level=0}{Expressions}{
Haskell expressions contain terms, operators and applications. Any function
name can be converted to an operator using backtics (e.g. \texttt{elem a b} to
\texttt{a `elem` b}), any operator can be converted to a prefixed function name using
parentheses (e.g. \texttt{1+2} to \texttt{(+) 1 2}).

Take a formula for roots of quadratic equation and convert all operators to
prefixes.

Which functions are better written in `infix` form?
}
\exercise{level=0}{Expression types}{
Haskell does not support automatic conversion of types: Find out why expression

\texttt{1/4+3}

works, but

\texttt{1/4+(3::Integer)}

fails with a type error. Fix the type error using a manual type conversion, so
that the literal 3 remains Integer.
}
