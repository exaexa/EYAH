\exercise{}{Expressions}{
Similarly as with backticks, any operator can be converted to a prefixed
function name using parentheses (e.g. \texttt{1+2} to \texttt{(+) 1 2}).

Take a formula for roots of quadratic equation and convert all operators to
prefixes.

Which functions are better written in \texttt{`infix`} form?
}
\exercise{}{Expression types}{
Haskell does not support automatic conversion of types: Find out why expression

\texttt{1/4+3}

works, but

\texttt{1/4+(3::Integer)}

fails with a type error. Fix the type error using a manual type conversion
(using function \texttt{fromInteger}), so that the literal 3 remains Integer.
}
\exercise{pencils=1}{Arguments}{
Using \texttt{:t}, attempt to print out the type of \texttt{min}, \texttt{min
1}, \texttt{min 1 2}, and \texttt{min 1 2 3}. Try to explain why only
\texttt{min 1 2} can be evaluated without an error message.
}
