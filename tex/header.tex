\documentclass[10pt,a5paper]{book}
\usepackage[utf8]{inputenc}
\usepackage[sc]{mathpazo}
\linespread{1.05}
\usepackage[T1]{fontenc}
\usepackage{marginnote}
\usepackage{wasysym}
\let\Square\undefined
\usepackage{bbding}
\usepackage{keyval,xparse}
\usepackage{multido}
\usepackage{hyperref}

\setlength{\marginparpush}{0pt}
\setlength{\marginparsep}{2em}

\title{Exercise Yourself a Haskell \\ \vspace{1em} \Large (For Great Profit)}
\date{}
%\author{People of the Internet}

%% Macro definitions here!

\newcommand{\startExercises}{\begin{description}}
\newcommand{\finishExercises}{\end{description}}

\makeatletter
\define@key{exerciseTag}{stars}{\def\exerciseTag@stars{#1}}
\define@key{exerciseTag}{lambdas}{\def\exerciseTag@lambdas{#1}}
\define@key{exerciseTag}{pencils}{\def\exerciseTag@pencils{#1}}
\define@key{exerciseTag}{trees}{\def\exerciseTag@trees{#1}}
\DeclareDocumentCommand{\exerciseTag}{m}{%
  \begingroup%
  % ========= KEY DEFAULTS + new ones =========
  \setkeys{exerciseTag}{stars={0},lambdas={0},pencils={0},trees={0},#1}%
  % First arg: \mm@first \par
  % Second arg: \mm@second \par
  % Third arg: \mm@third \par
  % Last arg: \mm@last
  \marginnote{
  \multido{}{\exerciseTag@stars}{\FiveStar}
  \multido{}{\exerciseTag@pencils}{\PencilLeftDown}
  \multido{}{\exerciseTag@lambdas}{{\large $\lambda$}}
  }[0pt]
  %\multido{\exerciseTag@lambdas}{$\mathbf{\lambda}$}
  \endgroup%
}
\makeatother
\newcommand{\exercise}[3]{\item[#2] \exerciseTag{#1} #3}

%% Start the main document. Write the book prologue down below.

\begin{document}
\sloppy

\maketitle

This book is intended as a collaborative project, everyone is welcome to
contribute at \url{https://github.com/exaexa/EYAH}.

For the reasons of great success, it is organized into chapters depending on
the way the reader is willing to exercise. First chapter is meant to accompany
the reading of the book `Learn Yourself a Haskell.' Second chapter are
collected exercises used on university course in middle Europe. We expect
parsing- and lens-oriented chapters to materialize.

Some exercises are marked by a variable number of badges:
\begin{itemize}
\item Stars \FiveStar\ represent an exercise that is worthy a bit of thinking.
\item Pencils \PencilLeftDown\ represent an exercise that attempts to improve
the technical background of the reader, at the cost of some writing or some
purely technical work with the source.
\item Lambdas {\large $\lambda$} mark some exercises that carry some reasonable
connection to underlying logic or $\lambda$-calculus.
\item (other tags to appear)
\end{itemize}

