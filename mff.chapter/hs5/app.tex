\exercise{level=3}{Apply-ish tree}{
Let's say we have a tree of unary functions and a tree of values from their
domain.

Construct tree of trees where each node in the overall tree corresponds to one
node in the function-tree, and contains a copy of the value-tree with all
values processed by the function.

Observe why binary trees do not make good \texttt{Applicative} instances.
}
\exercise{level=3}{Simple applicatives}{
Write and examine \texttt{Applicative} instance for \texttt{Maybe} (where
texttt{Nothing} applied to anything is again nothing) and \texttt{Either}
(where \texttt{Left} works as \texttt{Nothing} with an explanation).
}
\exercise{level=4}{Symmetric Either}{
Standard \texttt{Applicative} instance for \texttt{Either} discards one `error
message' if both parameters of \texttt{(<*>)} are \texttt{Left}.

Find a good reason for that. Specifically, what would happen if we would also
collect the error from the right side of \texttt{(<*>)}?
}
