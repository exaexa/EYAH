\exercise{}{And then?}{
Let's have a lookup function that finds something in an associative list, if it is there:

\texttt{lookup1 :: Eq k => k -> [(k,v)] -> Maybe v}

Implement a function that takes two keys, and returns a value at the position
\texttt{lookup1 k1 l + k2} (or Nothing if any of the required indices is not
present).

Implement the same function for three keys as \texttt{lookup3} (with the
difference that the \texttt{k3} is subtracted from the previous result to get
the next index), and for four keys as \texttt{lookup4} (\texttt{k4} is
multiplied with the previous result to get the index).

Find a repetitive pattern.
}
\exercise{pencils=1}{Aaand theeen?}{
Implement function that can take optional value on the left, and either pass
its value to function on the right and return the result, or, in case of
missing value, return the missing value:

\texttt{andThen :: Maybe a -> (a -> Maybe b) -> Maybe b}
}
\exercise{pencils=1}{And... Then?}{
Use construction

\texttt{computation1 `andThen` \textbackslash result1 -> computation2}

to reimplement the \texttt{lookup4} function.
}
\exercise{pencils=1}{And Then?!}{
Replace \texttt{andThen} in previous exercise by \texttt{>{}>=} and explain the
result.
}
\exercise{lambdas=1}{And Finally.}{
Use the \texttt{do}-notation to remove lambdas from the previous exercise.
}
\exercise{pencils=1}{MaybeM}{
Reimplement \texttt{Monad} instance of \texttt{Maybe}.
}
