\exercise{pencils=2}{Random state}{%
Implement a monad that keeps random number generator state. Use e.g. the
generator from \texttt{man 3 rand} (see \autoref{fig:man3rand}).

\begin{itemize}
\item Represent every random computation as a function from initial internal
state of the random number generator to a new internal state and the result of
computation:

\texttt{data RandM a = RandM (Int->(a,Int))}

\item Define a function \texttt{runRand} which executes the computation in
\texttt{RandM a} with some user-supplied seed, and extracts and returns the
result of type \texttt{a}.

\item Define a \texttt{RandM} computation called \texttt{rand} that takes a
seed and returns a generated random value and an updated seed.

\item Define a function \texttt{randSeq} that takes two \texttt{RandM}
computations and chains them --- constructs a new \texttt{RandM} which, given a
seed, runs the first \texttt{RandM} and feeds the resulting seed to the second
\texttt{RandM}.
\end{itemize}
}
\begin{figure}
\centering
\begin{minipage}{.9\textwidth}
\begin{verbatim}
int myrand(void) {
    next = next * 1103515245 + 12345;
    return((unsigned)(next/65536) % 32768);
}
\end{verbatim}
\end{minipage}
\caption{A relatively successful random number generator.}
\label{fig:man3rand}
\end{figure}

\exercise{lambdas=1,stars=1}{Random bind}{%
Similarly as with \texttt{randSeq} from the previous exercise, implement:

\texttt{randThen :: RandM a -> (a -> RandM b) -> RandM b}

The function passes the `return value' of the first computation as a parameter
to the function, which in turn returns second \texttt{RandM} computation. The
second computation receives receives the updated seed from the first
computation, to simulate that the computations are sequenced.

Observe the similarity with \texttt{andThen}.
}
\exercise{}{Random return}{%
Write a function \texttt{randRet} that takes a value and returns a random
computation which does not change the random seed, but returns the given value.
}
\exercise{lambdas=1}{Random monad}{%
Using \texttt{randThen} and \texttt{randRet}, define the \texttt{Monad} instance of \texttt{RandM}.
}
\exercise{stars=1}{Compute $\pi$}{%
Using your new monad for generating random numbers, establish the value of
$\pi$. Preferred method is to throw a pile of small stones on a square area
with inscribed circle, and count the ratio of stones that landed inside of the
circle.
}
\exercise{lambdas=1,pencils=1}{State}{
The RandM monad stores a single integer with state. We can generalize the concept to any type of the state:

\texttt{data StateM s a = StateChange (s -> (a,s))}

Write a monad instance for \texttt{StateM s}, together with appropriate
functions for getting and setting the state, and running the \texttt{StateM}.
}
