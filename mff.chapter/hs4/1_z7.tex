\exercise{}{Small cyclic groups}{
Take \texttt{data Z7 = Z7 Int deriving Show}. Define instance \texttt{Num Z7}
to allow using standard operators with elements $Z_7$.\footnote{In $Z_7$, all
operations behave like on integers modulo 7.}
}


\exercise{}{Smaller cyclic groups}{
If you haven't already, redefine the whole instance \texttt{Num Z7} as
point-free: That is best done by using a `lifting' function that takes an
operator/function that works on integers and converts it to work on $Z_7$.

See an excerpt from the target instance definition in \cref{fig:z7instance}.
}
\begin{figure}
\centering
\begin{minipage}{.6\textwidth}
\begin{verbatim}
instance Num Z7 where
  (+) = z7lift2 (+)
  (*) = z7lift2 (*)
  negate = z7lift1 negate
  ...
\end{verbatim}\end{minipage}
\caption{A foolproof way to lift the operators.}
\label{fig:z7instance}
\end{figure}

\exercise{stars=1}{$Z_7$ is not very universal}{
Find some reasons reasons why \texttt{Z7} should not be generalized (using a
type parameter for the internal integer type) to a \texttt{Functor}: one
obvious type-theoretic, and, after ignoring the first, one algebraic.
}
