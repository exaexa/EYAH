\exercise{}{filter2}{%
Function \texttt{filter} removes elements that do not match certain condition
from a list. Using it, write a similar function \texttt{filter2} that returns a
pair of lists --- first list contain elements that satisfy the condition,
second list contains the rest of the elements.
}
\exercise{}{filter2+}{%
If you have used two \texttt{filter} `calls' in the previous function, try
rewriting \texttt{filter2} not to use \texttt{filter}, and, specifically, to
only evaluate the condition \emph{once} for each element of the list.
}
\exercise{pencils=1}{Run-Length encoding}{%
Write a function that converts a list to run-length encoded form --- this is a kind of compression if you know that the list contains large subsequences of equal elements. For example $[1,1,1,2,5,5,5,3,3,3,3]$ should be converted to $[(1,3),(2,1),(5,3),(3,4)]$.
}
\exercise{}{RLE 2}{%
Check that your RLE function works on infinite lists (producing infinite compressed lists, of course):

\texttt{take 10 \$ rle \$ repeat [1,1,1,1,4,4]}

If not, fix it.
}
\exercise{stars=1}{RLE 3}{%
You can \texttt{import Data.List} to use the function \texttt{group}. Using it,
reimplement your RLE function as a oneliner.
}
