\section{Introductory lesson}

On the first Haskell lesson, students are shown how to run Haskell programs on
Windows and Linux. We use a setup similar to Prolog: An editor with open source
module that the students modify during the lesson, and next to it a terminal
with running \texttt{ghci} where the module is loaded using
\texttt{:l~themodulefile} for testing the features.

At this point, students usually know how to use Prolog (which saves them a lot
of difficulty when pattern-matching lists and types) and have seen a bit of
Scheme (which saves them another lot of difficulty after being shocked by how
simple the higher-level function can be).

At the first lesson, all code written on blackboard is annotated with the
`invisible' application operators between functions and their arguments. Reader
is suggested to do the same, learning the rule that when there is no operator
between expressions, there is instead a function application.
