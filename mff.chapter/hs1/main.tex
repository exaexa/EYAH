\exercise{level=0}{Simple I/O}{
Put together a haskell program with function \texttt{main} that reads a line of
text and writes it back to the user, and compile it using \texttt{ghc}. Use
functions \texttt{getLine} and \texttt{putStrLn}.
}
\exercise{level=0}{Formatted output}{
Print out some list of elements. Because the list is not a string, you will
need \texttt{show} that converts some types to a string. \footnote{Combination of \texttt{putStrLn} and \texttt{show} is called \texttt{print}.}
}
\exercise{level=0}{Formatted input}{
Just as \texttt{show} converts `anything' to String, \texttt{read} converts String to anything. Use it to read a Haskell-formatted list from input (for feeding it into some other function from this lesson).

How does \texttt{read} know what type it should return?
}
\exercise{level=1}{Computer thinks (a number)}{
Write a program where the computer thinks a number, user guesses it, and
computer tells him whether he guessed just right, too high or too low.

Use \texttt{r <- getStdRandom (randomR (1,10))} to get random values.
}
\exercise{level=1}{Humans can cheat}{
Write the same program in the opposite direction --- the user thinks a number
and computer guesses it. Include an angry reaction of the computer if the human
attempts to cheat.
}
