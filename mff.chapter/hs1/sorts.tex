\exercise{pencils=1}{Quicksort}{
You can easily generate functions from operators: For example, \texttt{(+1)} is
a function that increments the argument, or \texttt{filter (< 10)} filters a
list by removing numbers greater or equal to 10. Use this together with
list-concatenating operator \texttt{++} to implement a variant of quicksort.
}
\exercise{}{Merge}{
Write a function \texttt{merge} that merges two sorted lists together, to form
a new sorted list.
}
\exercise{}{Listify}{
For purposes of implementing mergesort, write a function that converts list of
items to a list of single-element lists that each contain one item from the
original list.
}
\exercise{}{Listify 2}{
If you didn't use \texttt{map} to implement the previous exercise, do it.
}
\exercise{}{Merge-reduce}{
Write a function that takes a list of lists and merges pairs of them using the
previously defined \texttt{merge} function, returning a list of lists. The
possible odd element on the end is unchanged.
}
\exercise{}{Mergesort}{
Using the previously defined function, implement mergesort: Split the list to
tiny lists, merge them until there is only one list left, and return it.
}
\exercise{}{Mergesort engineering}{
Fix the corner case with an empty list in your mergesort implementation using
the least amount of code possible.
}
