\exercise{level=5}{Kalotan puzzle (Kibi lied!)}{
The idea for this exercise is taken from `Teach Yourself Scheme in Fixnum Days':

\begin{quote}
The Kalotans are a tribe with a peculiar quirk. Their males always tell the
truth. Their females never make two consecutive true statements, or two
consecutive untrue statements.

An anthropologist (let's call him Worf) has begun to study them. Worf does not
yet know the Kalotan language. One day, he meets a Kalotan (heterosexual)
couple and their child Kibi. Worf asks Kibi: ``Are you a boy?'' Kibi answers in
Kalotan, which of course Worf doesn't understand.

Worf turns to the parents (who know English) for explanation. One of them says:
``Kibi said: `I am a boy.' '' The other adds: ``Kibi is a girl. Kibi lied.''

Solve for the sex of the parents and Kibi.
\end{quote}

While this makes a brilliant exercise for Prolog and a very interesting venue
for \texttt{call/cc} in Scheme, Haskell does not have very good facilities for
dealing with this kind of exercise directly.

Nevertheless, nondeterminism can be provided by several different types of
monads. Use any of them to simulate Prolog-ish behavior in Haskell.

Atop that, create reasonable functions that allow you to describe who said what
(including lists of statements!) and whether it is true or not. Attempt to
compress the `final' problem description to a piece of code as condensed and
semantic as possible. Try matching the simplicity of Prolog code in
\autoref{fig:kalotanProlog}.
}
\begin{figure}
\centering
\begin{minipage}{\textwidth}
\begin{verbatim}
solve(Kibi, P1, P2) :-
  % create a world
  truthval(KibiLies),
  sex(Kibi),
  sex(P1),
  sex(P2),
  not(P1 = P2),
  
  % says(?Who,?ListOfSentences,?TruthVals)
  % parent 1 says that Kibi says that
  % Kibi is male (also see whether Kibi lied)
  says(P1,
       [says(Kibi, [Kibi=male],
                   [KibiLies])], _),

  % parent 2 says that Kibi is female
  % and that Kibi lies
  says(P2,
       [Kibi=female, KibiLies=yes], _).
\end{verbatim}
\end{minipage}\caption{Final piece of Kalotan puzzle solution in Prolog.}
\label{fig:kalotanProlog}
\end{figure}
