\exercise{pencils=2}{Blackjack (lumber edition)}{
You have a shuffled deck of cards represented as a list with their numerical
values (e.g. [1,2,3,...,10,10,10]). Assume you play a following (dumb) variant
of Blackjack: On each turn you take 1 card and decide on next action to make
sure you hit total score of 21 as close as possible, but never go above that
score.  Your choices consist of `hitting', which means taking another card to
hand, possibly getting closer to 21, and `sticking', where you decide you are
close enough to 21 and compete with the (invisible) dealer.

Scoring is as such: A victory happens if the player hits 21 (automatically) or
if he `sticks' and his sum is higher than the sum of next 2 cards in the deck
(these are discarded; if there aren't enough cards in the deck, less is taken).
Victory increases the score by +2 points, loss (overshooting 21 or sticking
with less score than that of the next 2 cards in the pack) decreases the score
by 1 point. After victory/loss you automatically receive next card (if
present).

Materialize a complete decision tree of the players action: In the tree, left
child node describes the situation after `hitting' next card, right child node
describes situation after `sticking'. In the elements, the tree should contain
total score of the player so far, cards he currently holds in the hand, and
cards that are left in the deck.
}
\exercise{}{Pseudorandom permutation}{
Assume you receive a sufficiently large pseudorandom number. Use it to generate
a corresponding pseudorandom permutaion of the list you receive.

As always, note that there is no randomness in computers.
}
\exercise{}{Blackjack (bad statistics edition)}{
Randomly permute the Blackjack `cards' several times to generate differently
scored decision trees and collect a statistic about final player scores. Find a
path through the decision tree that would score best on average.
}
\exercise{}{Blackjack (quitter edition)}{
Player can quit the game whenever he wants, even before finishing the deck,
taking the score he has as final. Find the sub-path to the statistically best
score.
}
\exercise{stars=1}{Blackjack (better statistics edition)}{
Convert the statistics from the tree to a table that, depending on what sum the
player currently has on hand, advises him whether it is better to hit a next
card or stick with what he has.
}
\exercise{stars=1}{Blackjack (not-a-normal-distribution edition)}{
Measure the optimal ratio between `hits' and `sticks'.
}
