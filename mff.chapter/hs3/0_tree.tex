\exercise{}{Tree-ish data}{
Find any good representation of a binary search tree with elements of given
type, and write \texttt{data} with its definition.
}
\exercise{}{Tree to list}{
Define a function that converts your tree to a list of elements (in
tree-traversing order).
}
\exercise{}{List to tree}{
Make a balanced tree out of a list. Note that there are two approaches to the balance:
\begin{itemize}
\item Exactly splitting the incoming list at the middle and recursing to subtrees, which requires knowing the length of the list.
\item Building the tree iteratively from the `lower left', which produces a slightly different kind of balance and does not require measuring the list before splitting it.
\end{itemize}
}
\exercise{stars=1}{List halving}{
If you already haven't, create a function that splits the list roughly in
half \emph{without} using any integers or \texttt{length}.
}
\exercise{}{Tree size}{
Write a function that measures the number of elements in your tree.
}
\exercise{}{Tree find}{
Write a function that returns a \texttt{Bool} that says whether a value is contained in a tree
}
\exercise{}{Traversing trees in order behaves like what?}{
Derive an instance of \texttt{Foldable} for your tree and check that
\texttt{foldr}, \texttt{length}, \texttt{elem} and similar functions that work
on lists also work on your tree now.
}
\exercise{stars=2}{Map tree?}{
Try to define a version of \texttt{map} called \texttt{mapTree} that would work
on your tree. Why the standard \texttt{map} does not work just with
\texttt{Foldable} instances?

You can compare your findings with \texttt{foldMap} definition.
}
\exercise{pencils=1}{$n$-th tree element}{
Extract $n$-th element of a tree (in the traversing order). If there is no such
element, return Nothing.

Compare the efficiency and failure modes of ``specialized'' implementation
(that recursively looks into the correct subtrees) with a
\texttt{Foldable}-derived version that uses \texttt{toList} and operator
\texttt{(!!)}.
}
\exercise{pencils=1}{$k$-th foldable}{
If you didn't do it yet, rewrite operator \texttt{(!!)} to work on
\texttt{Foldable}. Convert your solution to a point-free one, preferably using
the composition operator.
}
